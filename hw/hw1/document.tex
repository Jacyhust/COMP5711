\documentclass[12pt,onecolumn,a4paper]{article}
%landscape:横排
\usepackage{geometry}
\usepackage[utf8]{inputenc}
\usepackage{CJKutf8}
\usepackage{indentfirst}
\usepackage{setspace}
\usepackage{amsmath}
\usepackage{mathrsfs}
\usepackage{amsfonts}
\usepackage{graphicx} 
\usepackage{natbib}
\usepackage{caption}
\usepackage{subfigure}
\usepackage{subfig}
\usepackage{color}
\usepackage{booktabs}

%\usepackage{times}

\usepackage[justification=centering]{caption}%图片标题居中

%\usepackage{algorithm}  
%\usepackage{algorithmicx}  
%\usepackage{algpseudocode} 

\usepackage[vlined,ruled]{algorithm2e}
\usepackage{algorithmicx}
\usepackage[noend]{algpseudocode}

%\floatname{algorithm}{Algorithm} 
\renewcommand{\algorithmicrequire}{\textbf{输入:}}  
\renewcommand{\algorithmicensure}{\textbf{输出:}}

\newtheorem{definition}{定义}[section]
\newtheorem{proof}{Proof}
\newtheorem{lemma}{Lemma}[section]
\newtheorem{question}{Question}[section]
\newtheorem{Symbol}{Symbol}[section]
%\setlength{\parindent}{2em}
\geometry{left=2cm,right=2cm,top=2.5cm,bottom=2.5cm}
\def\cmt{\textcolor{blue}}
\def\todo{\textcolor{red}}
\def\cbl{\textcolor{blue}}
\newcommand{\DD}{\mathrm{\mathcal{D}}}
\newcommand{\HH}{\mathrm{\mathcal{H}}}
\newcommand{\aka}{\emph{a.k.a.}\xspace}

\title{\textbf{COMP 5711 - Advanced Algorithms} \ \textbf{Written Assignment \# 1}}
\author{ZHAO, Xi \\12256508 \\xzhaoca@connect.ust.hk}
%\date{Nov 2018}
\begin{document}
\maketitle

\section*{Fixed Parameter Algorithms (KT Ch 10)}
\subsection*{Problem 1}
\textbf{Algorithm:} Let $B_i = \{x_1, \dots, x_c\}$ be any of the given sets in the instance. If $\mathcal{H}$ is a $k$-size hitting set, then at least one element in $B_i$ must be in $\mathcal{H}$. If $x_i \in \mathcal{H}$, we can remove all sets $B$ that contain $x_i$ and then find a $(k-1)$-size hitting set for the remaining sets. We can consider all possible cases where different $x_i$ belong to $\mathcal{H}$. Since there are a total of $c$ different $x_i$, the $k$-size hitting set problem can be reduced to $c$ $(k-1)$-size hitting set problems. Repeatedly, we solve the problem by reducing it to a $1$-size hitting set problem. Such $x_i$s constitute the final result. If there is a legal result found by this procedure, we return \textit{yes}; otherwise, return \textit{no}.

\textbf{Correctness:} If the algorithm finds a result, the result must be a legal $k$-size hitting set. If a $k$-size hitting set $\mathcal{H}$ exists, it can be found by this process because, for any set $B_i$, $B_i$ has at least one element in $\mathcal{H}$.

\textbf{Running Time:} Let $T(k, m)$ be the running time of such a $k$-size hitting set problem with $m$ instances; we have:

\begin{equation}
	T(k, m) = cT(k-1, m-m_i) + m\log c
\end{equation}
where $m_i$ is the number of sets containing $x_i$ and $m\log c$ is the time to check which sets contain $x_i$. So, $T(k, m) \leq cT(k-1, m) + m\log c$ and $T(k, m) = O(c^k km \log c)$. Therefore, $f(c, k) = c^k k \log c$ and $p(m) = m$.

\subsection*{Problem 7}
\subsubsection*{(a)} 
Let $T$ be such a tree decomposition of G with tree width $w$, and let $X$ be a bag in $T$. Then, the induced subgraph of $G$ on $X$ and its neighbors in $T$ is a tree with at most $w + 1$ vertices.
We can color this tree with at most $w + 1$ colors using a greedy algorithm. Start at the root of the tree and assign the first color to the vertices in $X$. Then, for each child of the root, assign the second color to the vertices in its bag that are not already colored. Continue this process for each level of the tree until all vertices are colored.

Now, we can extend this coloring to the entire graph $G$. For each bag $X$ in $T$, assign to each vertex in $X$ the minimum color that is not used by any of its neighbors in $T$. This is a valid coloring of $G$ with at most $w + 1$ colors, since each vertex has at most $w + 1$ neighbors in $T$.

Therefore, the chromatic number of $G$ is at most $w + 1$.

\subsubsection*{(b)}
We determine whether $G$ is $k$-colorable for $k=1,2,\dots,w+1$ by the dynamic programming. Let $(T, \{V_t: t\in T\})$ be a tree decomposition of $G$. 
For the subtree rooted at $t$, and every coloring $X$ of $V$ using the color set $(1,2,..., k)$, we use $f_t(X)$ to indicate whether there is a $k$-coloring of $G_t$ by using $X$ when restricted to $V_t$. 
We compute the values $g_t(x)$ when $t$ is a leaf by simply trying all possible colorings of $G_t$. In general, suppose $t$ has children $t_1,...,t_d$, and we know the values of $g_{t_i}(X)$ for each choice of $t_i$ and $X$. Then there is a coloring of $G_t$ consistent with $X$ on $V$ if and only if there are colorings of the subgraphs $G_{t_1},..., G_{t_d}$ that are consistent with $X$ on the parts of $V_{t_i}$ that intersect with $V_t$. Thus, $g_t(X)$ is true if and only if there are colorings $X_i$ of $V_{t_i}$ such that $g_{t_i}(X_i)$ and $X_i$ is the same as X when restricted to $V_t\cap V_{t_i}$.
This requires us to maintain $k^{(w+1)} < (w + 1)^{(w+1)}$ values for each piece of the tree decomposition.
\section*{Approximation Algorithms (KT Ch 11)}
\subsection*{Problem 7}
\subsubsection*{(a)} 
We process the customers in an arbitrary order. At any given point in time, let $V_j$ denote the total value of all customers who have been shown ad $j$. As we see each new customer, we show him the ad for which $V_j$ is the smallest.
Let $s'$ denote the spread of this algorithm. 
Suppose that ad $j$ is the one achieving the minimal spread. Let $i$ be any other ad and $c$ be the last customer to be shown ad $i$. Before $c$ was shown ad $i$, the value of $V_i$ was at most $V_j$ (by the definition of our greedy algorithm), and so $V_i \le V_j + v_c < V + \bar{v}/(2m)$, then
$$\bar{v} = \sum_{k=1}^{m} V_k < V_j + (m-1)(V_j + \bar{v}/(2m)).$$
Hence, $V_j > \bar{v}/(2m)$ by solving the inequality.

\subsubsection*{(b)} 
Assume that $n=9$ and $m=2$, $8$ customers have the same value $v_i=l$ and a customer $c_j$ has the value $2$. Then our greedy algorithm will produce a spread of $4$ when $c_j$ appears at the last step, while the optimal spread is $5$.

\subsection*{Problem 10}
\subsubsection*{(a)} 
If $v \notin S$, it must have been deleted in some iteration by the selection of a node $v'$. This node $v'$ must be a neighbor of v. Since $v'$ rather than $v$ is chosen, $v'$ must have at least as much weight as $v$.
\subsubsection*{(b)} 
Consider any other independent set $T$. For each node $v \in T$, if $ v\in S$, then we charge $w$ to itself. Otherwise, $v$ is a neighbor of some node $ v'\in S$ whose weight is at least as large as $v$. We charge $v$ to $v'$. Now, if $v$ is charged to itself, then no other node is charged to $v$, since $S$ and $T$ are independent sets. Otherwise, at most four neighboring nodes of no greater weight are charged to $u$. Either way, the total weight of all nodes charged to $v$ is at most $4w(u)$. Since these charges account for the total weight of $T$, it follows that the total weight of nodes in $T$ is at most four times the total weight of nodes in $S$.
\subsection*{Problem 11}
We define a normalized weight $w_i' = \left\lfloor \frac{w_i}{\epsilon W/n}\right\rfloor \cdot n/\epsilon W$ for each item $i$. For those items whose weight exceeds $W$, they cannot be put in $O$ and we remove them. Then, we find the solution $\mathcal{A}$ by the dynamic programming algorithm for the knapsack problem with small normalized weights. Since $w' \le W' = n/\epsilon$, the total cost is $O(nW') = O(n^2 / \epsilon)$.
Next, we prove that the method can find the correct solution. We can find the subset of normalized weight at most $W$ which achieves the greatest value $V'$. $V'$ must be greater than $V$ because the total sum of normalized weights in $O$ is at most $W$. When we put all these items in our knapsack, each has a weight that may be up to $\epsilon W/n$ more than we thought due to rounding error, so we use a weight of at most $ W + n(\epsilon W/n)=W(1+\epsilon)$.

\end{document}

