\documentclass[12pt,onecolumn,a4paper]{article}
%landscape:横排
\usepackage{geometry}
\usepackage[utf8]{inputenc}
\usepackage{CJKutf8}
\usepackage{indentfirst}
\usepackage{setspace}
\usepackage{amsmath}
\usepackage{mathrsfs}
\usepackage{amsfonts}
\usepackage{graphicx} 
\usepackage{natbib}
\usepackage{caption}
\usepackage{subfigure}
\usepackage{color}
\usepackage{booktabs}

%\usepackage{times}

\usepackage[justification=centering]{caption}%图片标题居中

%\usepackage{algorithm}  
%\usepackage{algorithmicx}  
%\usepackage{algpseudocode} 

\usepackage[vlined,ruled]{algorithm2e}
\usepackage{algorithmicx}
\usepackage[noend]{algpseudocode}
\usepackage{hyperref}
%\hypersetup{
%	colorlinks=true,
%	linkcolor=cyan,
%	filecolor=blue,      
%	urlcolor=red,
%	citecolor=green,
%}

%\floatname{algorithm}{Algorithm} 
\renewcommand{\algorithmicrequire}{\textbf{输入:}}  
\renewcommand{\algorithmicensure}{\textbf{输出:}}

\newtheorem{definition}{定义}[section]
\newtheorem{proof}{Proof}
\newtheorem{lemma}{Lemma}[section]
\newtheorem{question}{Question}[section]
\newtheorem{Symbol}{Symbol}[section]
%\setlength{\parindent}{2em}
\geometry{left=2cm,right=2cm,top=2.5cm,bottom=2.5cm}
\def\cmt{\textcolor{blue}}
\def\todo{\textcolor{red}}
\def\cbl{\textcolor{blue}}
\newcommand{\DD}{\mathrm{\mathcal{D}}}
\newcommand{\HH}{\mathrm{\mathcal{H}}}
\newcommand{\aka}{\emph{a.k.a.}\xspace}
\DeclareMathOperator{\E}{E}
\title{\textbf{COMP 5711 - Advanced Algorithms} \ \textbf{Written Assignment \# 4}}
\author{ZHAO, Xi \\12256508 \\xzhaoca@connect.ust.hk}
%\date{Nov 2018}
\begin{document}
\maketitle

%\section*{Amortized Analysis (CLRS Ch 17)}
\subsection*{SKI}
\subsubsection*{(a)}
We adopt the break-even algorithm \footnote{\url{https://en.wikipedia.org/wiki/Ski_rental_problem}} to solve this problem: 
I rent for $n-1$ days and buy skis on the morning of day $n$ if I am still up for skiing. If I have to stop skiing during the first $n-1$ days, it costs the same as what I would pay if one had known the number of days I would go skiing. If I have to stop skiing after day $n$, my cost is $(2n-1)$ dollars, which is at most as twice as what I would pay if I had known the number of days I would go skiing in advance. So, it achieves a competitive ratio of at most $2$.

\subsubsection*{(b)}
Consider an adversary that knows the deterministic strategy and chooses the number of days to ski accordingly. If the strategy always rents for $k$ days before buying, the adversary can choose to ski for $k+1$ days and then stop, forcing the strategy to rent for $k$ days and then buy a ski, which costs $k+n$ dollars. When $k<n-1$, \textsf{OPT}$=k+1$, 
$$\frac{k+n}{k+1}=1+\frac{n-1}{k+1}>1+\frac{n-1}{n}=2-1/n.$$
When $k\ge n-1$, \textsf{OPT}$=n$, 
$$\frac{k+n}{n}=1+\frac{k}{n}\ge 2-1/n.$$

So, no deterministic strategy can achieve a competitive ratio better than $2-1/n$.

\subsubsection*{(c)}
Since $n$ is an even number, let $n=2k$ and $q=1-1/k (q<1)$. We rent the ski until day $n/2=k$. Then with the probability $p_i=\alpha q^{2k-i}$ we buy the ski on the morning of day $i$ if I am still up for skiing; otherwise we continue to rent until day $i$.
Here $\sum_{i=k+1}^{2k} p_i=1$ and thus 
\begin{equation*}
	\alpha=\frac{1}{\sum_{i=k+1}^{2k} q^{2k-i}}=\frac{1-q}{1-q^k}.
\end{equation*}
When we buy the ski in the day $i$, the competitive ratio is at most $\frac{n+i-1}{n}<1+i/n$ based on the results in (a) and (b). Therefore, the competitive ratio in expectation is at most
\begin{equation*}
	\begin{split}
		E<&\sum_{i=k+1}^{2k} (1+i/n)p_i\\
		=&3/2+\frac{1-q}{2k(1-q^k)}\cdot\frac{kq}{1-q}\\
		=&3/2+\frac{q}{2(1-q^k)}\\
		<&3/2+q/2=2-1/n.
	\end{split}
\end{equation*}
So, we obtain a better competitive ratio.
%\begin{equation*}
%	E=\sum_{i=k+1}^{2k} (1+i/n)p_i=
%\end{equation*}
\subsection*{MG}
\subsubsection*{(a)}
After counting $N$ items, there are at most  $N-M$ decrements. So, at most $(N-M)/(k+1)$ decrements on each unique key, indicating the error is at most $(N-M)/(k+1)$.
\subsubsection*{(b)}
Denote $f_i'$ as the counter value of the i-th most frequent item . We have $f_i'\ge f_i-(N-M)/(k+1),1\ge i\ge t$. So,
\begin{equation}\label{eq:1}
	M\ge \sum_{i=1}^{t} f_i'\ge \sum_{i=1}^{t} (f_i-\frac{N-M}{k+1})=\sum_{i=1}^{t} f_i-\frac{t(N-M)}{k+1}
\end{equation}
According to Eq. \ref{eq:1},
\begin{equation}\label{eq:2}
	N-M\le N-\sum_{i=1}^{t} f_i+\frac{t(N-M)}{k+1}=N^{res(t)}+\frac{t(N-M)}{k+1}.
\end{equation}
So, $(N-M)/(k+1)\le N^{res(t)}/(k+1-t)$. Proven!
\subsection*{Parallel}
%We add an element $1$ in the end of $A$ to avoid that there is no $1$ in $A$ and denote the 
Since we can obtain the location of the first 1 by comparing each element once in $A$, where each comparison takes $O(1)$ times. Therefore, like the prefix sum problem, this problem can be simulated by a circuit of depth $D=O(\log n)$ and work $O(n)$. When there are $p=O(n)$ processors, according to the Brent's theorem, the running time can be $O(W/p+D)=O(\log n)$.

\end{document}

